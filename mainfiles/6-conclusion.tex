%!TEX root = ../diploma.tex

\section{Заключение}

В настоящей работе были рассмотрены задача оптимизации используемого объема
памяти, а также задача параллельных вычислений нейронной сети. Предложен метод
оптимизации на уровне ациклического графа вычислений, который совмещает
эвристический алгоритм решения задачи оптимального распределения памяти с
параллелизацией вычислений вершин графа. Его параметризация задает различные
соотношения между потреблением времени, памяти и электрической энергии при
вычислении НС на мобильном устройстве. Это позволяет разработчику выбрать
компромисс, исходя из результатов измерений для различных конфигураций метода.

Предложенный метод был реализован в программно-аппаратной части библиотеки
искусственного интеллекта MindSpore Lite на языке C++ с использованием
интерфейса прикладного программирования Vulkan. Реализация протестирована на
работоспособность и корректность на различных существующих НС.

Проведены эксперименты с использованием мобильного устройства с интегрированным
графическим процессором, которые показывают применимость предложенного метода в
том смысле, что он позволяет достигать сокращения потребляемого объема памяти и
при этом уменьшать время вычисления НС. В частности, показано, что нет единой
конфигурации параметров метода, позволяющей достигать оптимальных затрат для
всех архитектур НС.

Результаты применения метода к сверточной НС с групповыми свертками
ResNeXt~\cite{resnext} показывают, что при правильном выборе числа групп, на
которые разбиваются слои графа, достигается ускорение на 6.85\% и сокращение
потребления памяти на 72.4\%. Вместе с этим эмпирически показано, что на время
вычисления предсказания НС существенное влияние могут оказывать, помимо
синхронизаций между слоями, особенности обращений в память ГП вычислительными
ядрами.

Таким образом, полностью выполнены задачи, поставленные в работе: исследованы
существующие методы оптимизации объема памяти и затраченного времени на
вычисление предсказания НС на уровне АГВ; разработан и реализован метод,
позволяющий одновременно оптимизировать и затраченное время, и объем
используемой памяти; эмпирически исследована применимость предложенного метода и
произведено его сравнение с другими.
