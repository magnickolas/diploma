%!TEX root = ../diploma.tex
\section{Введение} 

В последние годы нейронные сети (НС) приобрели большую популярность в
применениях машинного обучения к различным прикладным задачам. Отчасти это
обусловлено приростом вычислительных мощностей графических процессоров (ГП):
современные НС часто обучают и запускают с использованием нескольких ГП,
вычисления происходят на облачных серверах и кластерах.

В приложениях на мобильных устройствах с использованием НС иногда возникает
потребность вычисления предсказаний НС непосредственно на самих мобильных
устройствах. Причин этому несколько, в частности: защита данных пользователя,
возможность использования приложения без доступа к сети Интернет, сокращение
затрат на сервера, уменьшение временной задержки на получение предсказания НС. 

Вместо использования центрального процессора мобильного устройства для
вычисления предсказаний НС более эффективно вычисление на его ГП, поскольку
многие операции в НС можно разбить на множество маленьких задач, решаемых
независимо друг от друга, что вполне соответствует архитектурным особенностям
ГП.  

При решении задач, связанных с классификацией и обработкой изображений и
видеопоследовательностей, активно используются разновидности идеи сверточных
нейронных сетей (СНС), предложенной в работе~\cite{cnn}. Также СНС оказались
эффективными и в решении задач из области обработки естественного
языка~\cite{conv_nlp},~\cite{cnn_text_class}. На примере сети
ResNeXt~\cite{resnext} было показано, что архитектурная реализация сверточных
слоев в СНС с помощью групповых сверток, изначально предложенных в
работе~\cite{alexnet}, улучшает точность предсказания, а также вычислительно
более эффективна по сравнению с классическими свертками.

Операции двумерной свертки в СНС вычислительно затратны и возникает естественное
желание добиться эффективности их вычисления. Для их ускорения при фиксированном
размере ядра используются вариации алгоритма вычисления сверток
Винограда~\cite{winograd1980arithmetic} — это позволяет ускорить вычисление
операции свертки в несколько раз.

В силу ограниченности ресурсов мобильного устройства также есть проблемы
потребления памяти и электрической энергии. Особенности обращений в память ГП
влияют на время их выполнения, что тоже сказывается на временных затратах.

Основной вклад данной работы заключается в следующем:
\begin{itemize}
    \item разработан метод оптимизации, позволяющий сокращать объем используемой
    памяти и общее время исполнения;
    \item предложенный метод реализован в программно-аппаратной части библиотеки
    искусственного интеллекта MindSpore Lite\footnote{Кодовая база
    \texttt{MindSpore}: \url{https://github.com/mindspore-ai/mindspore}.} на
    языке C++ с использованием интерфейса прикладного программирования Vulkan;
    \item произведено эмпирическое исследование предложенного метода и его
    сравнение с базовой реализацией и подходами из
    работ~\cite{pisarchyk2020efficient},~\cite{node_level_par}.
\end{itemize}

Настоящая работа организована следующим образом. В разделе~\ref{sec:task}
формально описываются задача оптимизации памяти и задача параллельных
вычислений. Обзор существующих подходов к их решению приводится в
разделе~\ref{sec:literature}. В разделе~\ref{sec:method} описывается
предложенный метод. Экспериментальное исследование метода и сравнение с другими
подходами приводится в разделе~\ref{sec:exp}.
